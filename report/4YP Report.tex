\documentclass[a4paper,oneside,11pt]{book}
\usepackage[hmargin=2cm,vmargin=3cm,bindingoffset=0.5cm]{geometry}


\usepackage{setspace}
\usepackage{graphicx}
\usepackage{subfigure}
\usepackage{amsfonts}
\usepackage{amsmath}
\usepackage[algoruled,vlined]{algorithm2e}
\usepackage{listings}
\usepackage{pdfpages}
\usepackage{fancyhdr}

\pdfminorversion=5
\pagestyle{headings}
%\linespread{1.5}
\doublespacing
\raggedbottom
\begin{document}

\frontmatter
\begin{titlepage}
\begin{center}
\includegraphics[width=50mm]{images/oxlogo.png}\\
\ \\ % force an empty line
\ \\
\textsc{\Large 
%
Department of Engineering Science\\
\ \\
Final Year Project\\
\ \\}
{\LARGE 
Fast Level Set Segmentation of Biomedical Images using Graphics Processing Units
\\}
\ \\
\ \\
\ \\
\ \\
\ \\
\begin{minipage}{0.4\textwidth}
\begin{flushleft} \large
\emph{Author:}\\
Hormuz~Mostofi\\
Keble College
\end{flushleft}
\end{minipage}
%
% right column
\begin{minipage}{0.4\textwidth}
\begin{flushright} \large 
\emph{Supervisors:} \\
Dr. Julia Schnabel\\
Dr. Vicente Grau
\end{flushright}
\end{minipage}

\vfill
{\large {May 2009}}
\end{center}
\end{titlepage}


\includepdf{images/declaration.pdf}

\section*{Acknowledgments} 
I would like to greatly thank my supervisors \textit{Dr Julia Schnabel} and \textit{Dr Vicente Grau} for their excellent guidance in this project. Their ability to point out the subtle technicalities of biomedical image and volume segmentation proved an invaluable resource. 

I would also like to thank \textit{Professor Mike Giles} and \textit{Daniel Goodman} for their guidance in CUDA programming. I found the weekly CUDA meetings led by \textit{Professor Mike Giles} a very helpful forum to learn advanced CUDA programming techniques, whilst keeping up with any developments with respect to CUDA. 

Finally, I would like to thank all the staff of the Oxford E-Research Centre for providing the facilities and excellent work environment for the development of this project.

\section*{Personal Challenges}
At the outset of this project, I had very limited knowledge in the areas of biomedical image segmentation and implementations thereof. A vast literature review of level set segmentation techniques was required in order to take the knowledge gained from C6 Medical Image Analysis course to engineer an initial implementation. As my knowledge of C was initially fairly basic, this first implementation was written in MATLAB using the Image Processing Toolbox.

In order to optimize the algorithm, new skills had to be learnt for the development of a faster sequential C-language algorithm and a much faster parallel algorithm. For the sequential C-language algorithm, extensive knowledge and experience with pointers and arrays needed to be acquired alongside learning OpenGL for visualization. Optimizing further, in order to engineer the significantly faster parallel algorithm, a significant amount of time needed to be spent learning and experimenting with the NVIDIA CUDA programming language and architecture with all its intricacies and subtlety. 

The final challenge was to extend to three dimensions, increasing the complexity of sequential and parallel algorithms greatly. Therefore throughout this project there has been a very steep learning curve to overcome in terms of software engineering skill and experience with segmentation algorithms.


\setcounter{tocdepth}{1}
\begin{spacing}{1.2}
\tableofcontents
\end{spacing}

\mainmatter
\chapter{Introduction}

\section{Image Segmentation}
Image segmentation is the task of splitting a digital image into one or more regions of interest. It is a fundamental problem in computer vision and many different methods, each with their own advantages and disadvantages, exist for the task. Image segmentation is a particularly difficult task for several reasons. Firstly, the ambiguous nature of splitting up images into objects of interest provides a trade off between making algorithms more generalized and having many user specified parameters. Secondly, imaging artificats such as noise, inhomogeneity, acquisition artifacts and low contrast, are very difficult to account for in segmentation algorithms without a high level of interactivity from the user. 

In this report, segmentation is discussed in a medical imaging context however the proposed algorithm could equally be used in general purpose segmentations. Segmented images are typically used as the input for applications such as classification, shape analysis and measurement. In medical image processing, segmented images are used for studying anatomical structures, diagnosis and assisting in surgical planning.

Image segmentation also encompasses three dimensional volume segmentations, which are slower to compute by several orders of magnitude. It should be noted that before such algorithms existed, segmentation of medical images was done by hand by experts. This was a very accurate, yet slow, process. These segmentations will form the gold standard with which to validate algorithmic segmentations.

In this report, the level set method is used for the purposes of segmentation. Their principal disadvantage is that they are relatively slow to compute, which provides the motivation for optimizing and accelerating such algorithms using graphics processing units (GPUs). Section \ref{levelsetmethod} discusses them in great detail.



\section{Parallel Processing}
The algorithms for processing level sets have vast parallelization potential. Section \ref{levelsetalgorithm} details the algorithms used to discretize the level set equation.
	\subsection{GPGPU}
General purpose computation on graphics processing units (GPGPU) is the technique of using graphics hardware to compute on applications typically handled by the central processing unit (CPU). Graphics cards over the past two decards have been required to render increasingly complex 3D scenes at high frame rates, which is in itself a highly parallelizable task computationally. 

Compared to a CPU, a GPU features many more transistors on the control path due to the lower number of control instructions required. Memory is optimized for throughput and not latency, with strict access patterns. It is not optimized for general purpose programs, and does not have the complex instruction sets, or branch control of the modern CPU. It should be noted however that CPUs are slowly being parallelized by featuring multiple cores on a single chip.

The advent of GPGPU programming came with programmable shader units that allowed 

	\subsection{CUDA}
Compute Unified Device Architecture, or CUDA, is NVIDIA's GPGPU technology that allows for programming of the GPU without any graphics knowledge. The C language model has at its core three key abstractions, from \cite{cuda}: a heirachy of thread groups. shared memories, and barrier synchronization. This breaks the task of parallelization into three sub problems, which allows for languaging expressivity when threads cooperate, and scalability when extended to multiple processor cores.

		\subsubsection{Framework}
CUDA uses extends C by allowing a programming to write \textit{kernels} that when invoked execute a thousands of lightweight identical threads in parallel. CUDA arranges these threads into a hierarchy of blocks and grids, as can be seen in Figure \ref{fig:cudathreads} allowing for runtime transparent scaling of code to different GPUs. The threads are identified by their location within the grid and block, making CUDA perfectly suited for tasks such as image processing where each threads is easily assigned to an individual pixel or voxel.

\begin{figure}[p]
	\centering
		\includegraphics[scale=0.4]{images/cudathreads.PNG}
		\caption{A grid of thread blocks. This figure taken from \cite{cuda}}
	\label{fig:cudathreads}
\end{figure}

When writing and optimizing complex parallel code in CUDA it is often found that threads may need to cooperate. The memory hierarchy of CUDA threads is shown in Figure \ref{fig:cudamemory}. Here it can be seen that each thread has access to: a per-thread private local memory, a per-block on-chip shared memory to share data between threads, and finally an off-chip global memory accessible to all threads. There are also constant and texture memory spaces accessible to all threads, however these are not utilised in this reports algorithm and so will not be discussed in any further detail.

\begin{figure}[p]
	\centering
		\includegraphics[scale=0.4]{images/cudamemory.PNG}
		\caption{The memory heirarchy of CUDA threads and blocks. This figure taken from \cite{cuda}}
	\label{fig:cudamemory}
\end{figure}

		\subsubsection{Performance Guidelines}
There are many techniques to optimize a parallel algorithm. Firstly, the optimum block and grid sizes should be used to ensure maximum 'occupancy'. Occupancy is the ratio of the number of active warps (32 parallel threads) to the maximum number of active warps supported by the GPU multiprocessor. To maximise efficiency, there is a trade off between making the occupancy very high, ensuring no multiprocessor is ever idle, and making it low enough to ensure no bank conflicts.

Secondly, one of the best ways in which to optimize the parallelization is through efficient shared memory usage. The global memory space is not cached and therefore has a much higher latency and lower bandwidth than on-chip shared memory. Therefore it is the aim of the programmer to minimise global memory accesses. From \cite{cuda}, it recommended that each thread in a block firstly loads data from global memory to shared memory, synchronizes with all other threads within the threadblock to ensure shared memory locations have been written to, processes the data, synchronizes again to ensure shared memory has been fully updated with results, and finally writes the results back to global memory.

Coalescence
\chapter{Method}

In this section, we introduce the level set method and dynamic implicit surfaces. Their role in segmentation is discussed having introduced having defined mathematical constructs such as signed distance transforms.

\section{Level Set Method}\label{levelsetmethod}
The level set method evolves a contour (in two dimensions) or a surface (in three dimensions) implicitly by manipulating a higher dimensional function, called the level set function $\phi(\textbf{x,t})$. The evolving contour or surface can be extracted from the zero level set $\Gamma(\textbf{x,t})=\left\{\phi(\textbf{x,t}) = \textbf{0}\right\}$. The advantage of using this method is that topological changes such as merging and splitting of the contour or surface are catered for implicitly, as can be seen below in Figure \ref{fig:levelsets}. The level set method, since its introduction by Osher and Sethian in \cite{oshersethian}, has seen widespread application in image processing, computer graphics (surface reconstructions) and physical simulation (particularly fluid simulation).

	\begin{figure}[h]
		\centering
			\includegraphics[scale=0.4]{images/levelsets.png}
		\caption{The relationship between the level set function (left) and contour (right) can be seen. It can be seen evolving the surface splits the contour.}
		\label{fig:levelsets}
	\end{figure}

The evolution of the contour or surface is governed by a level set equation. The solution tended to by this partial differential equation is computed iteratively by updating $\phi$ at each time interval. The general form of the level set equation is shown below.

	\begin{equation}
	\frac{\partial{\phi}}{\partial{t}}=-|\nabla{\phi|}\cdot F
	\label{eq:levelsetequation}
	\end{equation}

In the above level set equation $F$ is the velocity term that describes the level set evolution. By manipulating $F$, we can converge the level set to different areas or shapes, given an initialisation of the level set function. 

\section{Segmentation using Level Sets}
Typically, for applications in image segmentation $F$ is dependent on the pixel intensity or curvature values. It may also be dependent on an edge indicator function, which is defined as having a value zero on an edge, and zero otherwise. This causes $F$ to slow the level set evolution when on an edge.

In \cite{Lefohn04astreaming} $F$ is dependent on a data term and a curvature term (with a weighting term between the two) for the purposes of image segmentation. Therefore, the level set equation takes the form

	\begin{equation}
	\frac{\partial{\phi}}{\partial{t}}=-|\nabla{\phi}|\left[\alpha D(\bar{x})  + (1-\alpha)\nabla \cdot{\frac{\nabla{\phi}}{|\nabla{\phi|}}}\right]
	\label{eq:fulllevelsetequation}
	\end{equation}

where the data function $D(I)$ tends the solution towards targeted features, and the mean curvature term $\nabla \cdot{(\nabla{\phi}/|\nabla{\phi|})}$ keeps the level set function smooth. Weighting between these two is $\alpha \in [0,1]$, a free parameter that is set beforehand to control how smooth the contour or surface should be.

The data function $D(I)$ acts as the principal 'force' that drives the segmentation. By making $D$ positive in desired regions or negative in undesired regions, the model will tend towards the segmentation sought after. A simple speed function that fulfills this purpose, used by Lefohn, Whitaker and Cates in \cite{Lefohn04astreaming, gist}, is given by

	\begin{equation}
	D(I)= \epsilon - |I-T|
	\label{eq:dataterm}
	\end{equation}

which is plotted in Figure \ref{fig:speedterm}. Here $T$ describes the central intensity value of the region to be segmented, and $\epsilon$ describes the intensity deviation around T that is part of the desired segmentation. Therefore if a pixel or voxel has an intensity value within the $T\pm\epsilon$ range the model will expand, and otherwise it will contract. 

	\begin{figure}[h]
		\centering
			\includegraphics[scale=0.3]{images/speedterm.png}
		\caption{The speed term from \cite{gist}}
		\label{fig:speedterm}
	\end{figure}

Therefore the three user parameters that need to be specified for segmentation are $T$,$\epsilon$ and $\alpha$. An initialization for the level set function is also required, which may take the form of a cube in three dimensions or a square in two dimensions, or any other arbitrary closed shape. 

PICTURE OF CURVATURE FAIL


	\subsection{Signed Distance Transform}
A distance transform assigns a value for every pixel (or voxel) within a binary image containing one or more objects the value of which represents the minimum distance from that pixel to the closest pixel on the boundary of the object(s). The mathematical definition of a distance function $D:\mathbb{R}^3 \rightarrow \mathbb{R}$ for a set $S$, from \cite{oshersethian}, is
	
	\begin{equation}
	D(r,S) = \textrm{min}{|r-S|} \textrm{ for all } r \in \mathbb{R}^3
	\label{eq:distancetransform}
	\end{equation}

A \textit{signed} distance transform assigns the sign of the distance value as positive for those pixels outside the object, and negative for those inside it. It should be noted that the distance values depend on the chosen metric for distance: some common distance metrics are Euclidean distance, chessboard distance, and city block distance. Many of the algorithms that compute signed distance transforms often trade accuracy for efficiency and feature varying levels of complexity.

Signed distance transforms are required to initialize $\phi$ and also to reinitialize it every certain number of iterations. Computation of the initialization of $\phi$ is required before iteration of the level set equation can take place, and this will typically be a signed distance transform of an initial mask. Therefore the level set segmentation filter requires two images: an initial mask (which indicates targeted regions) and a \textit{feature} image (which is the image to be segmented). 
	
	\begin{figure}
	  \begin{center}
	    \subfigure[Arbitrary Initial Mask]{\label{fig:initmask}\includegraphics[scale=0.5]{images/initmask.png}}
	    \subfigure[Signed Distance Transform of Mask with Zero Level Set Overlaid]{\label{fig:sdf}\includegraphics[scale=0.5]{images/sdf.png}}
	  \end{center}
	  \caption{2D Signed Euclidean Distance Transform}
	  \label{fig:sdfexplanation}
	\end{figure}



\chapter{Implementation}

\section{Level Set Algorithm}\label{levelsetalgorithm}

\subsection{Upwinding}
Equation \eqref{eq:levelsetequation}, the level set equation, needs to be discretized for both sequential and parallel computation. This is done using the \textit{up-wind} differencing scheme. The following explanation of \textit{upwinding} is from \cite{osher2003lsm}.

A first order accurate method for time discretization of equation \eqref{eq:levelsetequation}, is given by the forward Euler method, from \cite{osher2003lsm}:

\begin{equation}
\frac{\phi^{t+\Delta t}-\phi^t}{\Delta t} +F^{t}\cdot{\nabla{\phi^{t}}} = 0
\label{eq:euler1}
\end{equation}

where $\phi^{t}$ represents the current values of $\phi$ at time $t$, $F^{t}$ represents the velocity field at time $t$, and  $\nabla{\phi^{t}}$ represents the values of the gradient of $\phi$ at time $t$. When computing the gradient, a great deal of care must be taken with regards to the spatial derivatives of $\phi$. This is best exemplified by considering the expanded form of equation \eqref{eq:euler1}.

\begin{equation}
\frac{\phi^{t+\Delta t}-\phi^t}{\Delta t} +u^{t}\phi_x^t+v^{t}\phi_y^t+w^{t}\phi_z^t = 0
\label{eq:euler2}
\end{equation}

For simplicity, consider the one dimensional form of equation \eqref{eq:euler2} at a specific grid point $x_i$ 

\begin{equation}
\frac{\phi^{t+\Delta t}-\phi^t}{\Delta t} +u_i^{t}(\phi_x)_i^t = 0
\label{eq:euler3}
\end{equation}

where $(\phi_x)_i$ is the spatial derivative of $\phi$ at $x_i$. The method of characteristics indicates whether to use a forward difference or backwards difference for $\phi$ based on the sign of $u_i$ at the point $x_i$. If $u_i > 0$, the values of $\phi$ are moving from left to right, and therefore backwards difference methods ($D_x^-$) should be used. Conversely, if $u_i<0$, forward difference methods ($D_x^+$) should be used to approximate $\phi_x$. It is this process of choosing which approximation for the spatial derivative of $\phi$ to use based on the sign of $u_i$ that is known as \textit{upwinding}. 

Extending this to three dimensions, from \cite{Lefohn04astreaming}, results in the derivatives below required for the level set equation update. 

\begin{eqnarray}
	D_x &=& (u_{i+1,j,k}-u_{i-1,j,k})/2 \nonumber\\
	D_y &=& (u_{i,j+1,k}-u_{i,j-1,k})/2 \nonumber\\
	D_z &=& (u_{i,j,k+1}-u_{i,j,k-1})/2 \nonumber\\
	D_x^+ &=& u_{i+1,j,k}-u_{i,j,k} \nonumber\\
	D_y^+ &=& u_{i,j+1,k}-u_{i,j,k} \nonumber\\
	D_z^+ &=& u_{i,j,k+1}-u_{i,j,k} \nonumber\\
	D_x^- &=& u_{i,j,k}-u_{i-1,j,k} \nonumber\\
	D_y^- &=& u_{i,j,k}-u_{i,j-1,k} \nonumber\\
	D_z^- &=& u_{i,j,k}-u_{i,j,k-1} \nonumber\\
\end{eqnarray}

$\nabla\phi$ is approximated using the upwind scheme.

\begin{eqnarray}
\nabla\phi_{\max} &=& \left[
  \begin{array}{ c }
     \sqrt{\max(D_x^+, 0)^2 + \max(-D_x^+,0)^2}  \\[2em]
     \sqrt{\max(D_y^+, 0)^2 + \max(-D_y^+,0)^2}  \\[2em]
     \sqrt{\max(D_z^+, 0)^2 + \max(-D_z^+,0)^2}  
  \end{array} \right] \\[2em]
\nabla\phi_{\min} &=& \left[
  \begin{array}{ c }
     \sqrt{\min(D_x^+, 0)^2 + \min(-D_x^+,0)^2}  \\[2em]
     \sqrt{\min(D_y^+, 0)^2 + \min(-D_y^+,0)^2}  \\[2em]
     \sqrt{\min(D_z^+, 0)^2 + \min(-D_z^+,0)^2} 
  \end{array} \right] 
\end{eqnarray}

Finally, depending on whether $F_{i,j,k} > 0$ or $F_{i,j,k} < 0$, $\nabla\phi$ is 

\begin{equation}
\nabla\phi = \left\{ 
\begin{array}{l l}
  ||\nabla\phi_{\max}||_2 & \quad \mbox{if $F_{i,j,k} > 0$}\\
  ||\nabla\phi_{\min}||_2 & \quad \mbox{if $F_{i,j,k} < 0$}\\ \end{array} \right.
\label{eq:finalchoice}
\end{equation}

\begin{equation}
\phi(t+\Delta t) =\phi(t) + \Delta t F|\nabla\phi|
\label{eq:phi}
\end{equation}

The speed term $F$, as discussed before, is based on the pixel intensity values and curvature values. 

\subsection{Curvature}
Curvature is computed based on the values of the current level set using the derivatives below. In two dimensions only the first two derivatives are required, alongside the derivatives defined previously. In three dimensions, all the derivatives below are required.

\begin{eqnarray}
	D_x^{+y} &=& (u_{i+1,j+1,k}-u_{i-1,j+1,k})/2 \nonumber\\
	D_x^{-y} &=& (u_{i+1,j-1,k}-u_{i-1,j-1,k})/2 \nonumber\\
	D_x^{+z} &=& (u_{i+1,j,k+1}-u_{i-1,j,k+1})/2 \nonumber\\
	D_x^{-z} &=& (u_{i+1,j,k-1}-u_{i-1,j,k-1})/2 \nonumber\\
	D_y^{+x} &=& (u_{i+1,j+1,k}-u_{i+1,j-1,k})/2 \nonumber\\
	D_y^{-x} &=& (u_{i-1,j+1,k}-u_{i-1,j-1,k})/2 \nonumber\\
	D_y^{+z} &=& (u_{i,j+1,k+1}-u_{i,j-1,k+1})/2 \nonumber\\
	D_y^{-z} &=& (u_{i,j+1,k-1}-u_{i,j-1,k-1})/2 \nonumber\\
	D_z^{+x} &=& (u_{i+1,j,k+1}-u_{i+1,j,k-1})/2 \nonumber\\
	D_z^{-x} &=& (u_{i-1,j,k+1}-u_{i-1,j,k-1})/2 \nonumber\\
	D_z^{+y} &=& (u_{i,j+1,k+1}-u_{i,j+1,k-1})/2 \nonumber\\
	D_z^{-y} &=& (u_{i,j-1,k+1}-u_{i,j-1,k-1})/2 \nonumber\\
\end{eqnarray}

Using the \textit{difference of normals} method from \cite{Lefohn04astreaming}, curvature is computed using the above derivates with the two normals $\textbf{n}^+$ and $\textbf{n}^-$.

\begin{eqnarray}
\textbf{n}^+ &=& \left[
  \begin{array}{ c }
     \frac{D_x^+}{\sqrt{(D_x^+)^2 + {\left(\frac{D_y^{+x}+D_y}{2}\right)}^2 +{\left(\frac{D_z^{+x}+D_z}{2}\right)}^2  }}  \\[2em]
     \frac{D_y^+}{\sqrt{(D_y^+)^2 + {\left(\frac{D_x^{+y}+D_x}{2}\right)}^2 +{\left(\frac{D_z^{+y}+D_z}{2}\right)}^2  }}  \\[2em]
     \frac{D_z^+}{\sqrt{(D_z^+)^2 + {\left(\frac{D_y^{+z}+D_x}{2}\right)}^2 +{\left(\frac{D_y^{+z}+D_y}{2}\right)}^2  }}  
  \end{array} \right] \\[2em]
\textbf{n}^- &=& \left[
  \begin{array}{ c }
     \frac{D_x^-}{\sqrt{(D_x^-)^2 + {\left(\frac{D_y^{-x}+D_y}{2}\right)}^2 +{\left(\frac{D_z^{-x}+D_z}{2}\right)}^2  }}  \\[2em]
     \frac{D_y^-}{\sqrt{(D_y^-)^2 + {\left(\frac{D_x^{-y}+D_x}{2}\right)}^2 +{\left(\frac{D_z^{-y}+D_z}{2}\right)}^2  }}  \\[2em]
     \frac{D_z^-}{\sqrt{(D_z^-)^2 + {\left(\frac{D_y^{-z}+D_x}{2}\right)}^2 +{\left(\frac{D_y^{-z}+D_y}{2}\right)}^2  }}  
  \end{array} \right] 
\label{eq:n}
\end{eqnarray}

The two normals are used to compute divergence, allowing for mean curvature to be computed as shown below in equation \eqref{eq:curv}.

\begin{equation}
H = \frac{1}{2}\nabla\cdot\frac{\nabla\phi}{|\nabla\phi|} = \frac{1}{2}((\textbf{n}_x^+ - \textbf{n}_x^-)+(\textbf{n}_y^+ - \textbf{n}_y^-)+(\textbf{n}_z^+ - \textbf{n}_z^-))
\label{eq:curv}
\end{equation}

\subsection{Stability}
From \cite{osher2003lsm}, a finite difference approximation to a linear partial differential equation is convergent if and only if it is both consistent and stable. Stability implies that small errors in the solution are not amplified during iteration. Stability is enforced using the Courant-Friedreichs-Lewy (CFL) condition which states the numerical wave speed must be greater than the physical wave speed, i.e. $\Delta x/\Delta t>|u|$. Rearranging, we have

\begin{equation}
\Delta t < \frac{\Delta x}{max\left\{|u|\right\}}
\label{eq:cfl}
\end{equation}

which is usually implemented, through variants of equation \eqref{eq:cfl}, by choosing a \textit{CFL number} that lies between 0 and 1 to further guarentee stability.

Another measure taken to ensure stability is the inclusion of a floating point relative accuracy term in the denominator of any fractions to avoid singularity errors as the denominator tends to zero. This is done in equations \eqref{eq:n} to ensure that $\textbf{n}$ does not tend to infinity if the square root is zero.

\section{Sequential Implementation}
Two dimensional implementations of the code in MATLAB, C and then CUDA were firstly written. Once these had been optimized, three dimensional implementations were written. 



	\subsection{Matlab}
The first task was to write code in MATLAB to segment two dimensional greyscale images. The MATLAB Image Processing Toolbox provides many functions (such as the ability to load, resample and filter images, compute distance transforms and easily visualise the level set evolution) which kept the code reasonably concise. 

The code is split into two files (a launcher and a kernel), in order to seperate the initialisation and level set update code. The user specifies parameters for threshold values $T$, range $\epsilon$ and curvature weighting $\alpha$, runs the launcher and then proceeds to draw a closed polygon that will form the initial mask (providing some basic interactivity). 

\begin{figure}[h]
	\centering
		\includegraphics[scale=0.6]{images/matlab.png}
	\caption{MATLAB user interface showing four subfigures with the input image, the initial mask, the current zero level set interface superimposed on the input image and the current level set surface in 3D}
	\label{fig:matlab}
\end{figure}

The level set function $\phi$ is then initialised to a signed distance function of this mask, and iteration of the level set equation begins for a fixed number of iterations (also user-definable). Reinitialisation of the level set is performed once every 50 iterations, and the current level set contour and surface are displayed every 20 iterations.

The derivatives are calculated by subtracting shifted matrices of $\phi$ from $\phi$ (or vice-versa). Note how derivatives are not calculated in an element by element fashion.

Finally, the user has the option of downsampling the input image in order to speed up the computation.



	\subsection{C}
Initially C code was written 


\section{Parallel Implemention}
	\subsection{Block and Grid Sizes}
	\subsection{Shared Memory}
\chapter{Results}

In the following, the results of the speed ups attained by optimizing using CUDA will be shown. However, before this can be done, some preliminaries need to be listed. Firstly, all hardware testing was done on a single PC with an Intel Core 2 Duo T8100 Processor with a clock speed of 2.1 GHz and 4 GB of RAM. The graphics hardware used was the NVIDIA GeForce 8600M GT, with CUDA 2.1 software installed. It should be noted that at the time of writing, CUDA 2.2 was available although in beta form and was not chosen due to potential instabilities. Timing code used was from the \texttt{cutil} library.

Although 8600M GT is adequate for CUDA development, it has rather limited performance in comparison to other graphics chips. This implies that the performance speed ups below could potentially be up to a further order of magnitude faster on newer hardware. For example, although the shader processing rate of 8600M GT is quoted as 91.2 Gigaflops the recently released GeForce GTX 295 boasts an impressive 1788.48 Gigaflops potentially allowing for another order of magnitude speed up from the 8600M GT hardware. This is mainly due to the increased number of on chip multiprocessors, however to lesser extent is also due to the device being of higher \textit{compute capability}: there are fewer limitations (such as support for double precision arithmetic) and relaxed requirements for coalescing memory transfers.

\section{Speed Tests and Analysis}\label{results}

\subsection{2D Segmentations}


In Figure \ref{fig:liver} the example of a liver segmentation is shown. The liver data is of good contrast and dimension $256\times 256$ (which is a multiple of 16 implying no memory padding is required in CUDA). The liver has been entirely segmented, with the initial mask as the input. The time taken for 5000 iterations in MATLAB, C, CUDA (Unoptimized) and CUDA (Optimzed) are shown in Table \ref{livertime}.

 
\begin{figure}[h]
  \begin{center}
    \subfigure[Feature Image Input - Liver]{\label{fig:liveroriginal}\includegraphics[scale=0.5]{images/liver.PNG}}
    \subfigure[Initial Mask Input]{\label{fig:livermask}\includegraphics[scale=0.5]{images/livermask.PNG}}
    \subfigure[Output of Segmentation]{\label{fig:liverseg}\includegraphics[scale=0.5]{images/liverseg.PNG}}
  \end{center}
  \caption{2D Liver Segmentation with parameters $T = 180, \epsilon = 45, \alpha = 0.003$}
  \label{fig:liver}
\end{figure}

\begin{table}[h]
\centering
\begin{tabular}{ | l | c | r | }
	\hline
	Algorithm Version  & Time (s)\\ \hline
  MATLAB 						 & 425.95 \\
  C 								 & 55.44 \\
  CUDA (Unoptimized) & 8.38 \\
  CUDA (Optimized)   & 1.73  \\
  \hline
\end{tabular}\caption{Comparison of runtime for different algorithm versions - 2D liver segmentation}
\label{livertime}
\end{table}

The runtime speed up attained from sequential code in C to CUDA optimized code is approximately $32 \times$. The block sizes used for 2D CUDA compute were $32 \times 8$ (the effect of varying block sizes in 3D is examined in \ref{blocksizes}.


\begin{figure}[h]
  \begin{center}
    \subfigure[Feature Image Input - brain]{\label{fig:brainoriginal}\includegraphics[scale=0.25]{images/brain.PNG}}
    \subfigure[Initial Mask Input]{\label{fig:brainmask}\includegraphics[scale=0.25]{images/brainmask.PNG}}
    \subfigure[Output of Segmentation]{\label{fig:brainseg}\includegraphics[scale=0.25]{images/brainseg.PNG}}
  \end{center}
  \caption{2D Brain Segmentation with parameters $T = 45, \epsilon = 30, \alpha = 0.003$}
  \label{fig:brain}
\end{figure}

In Figure \ref{fig:brain} we can see the brain in sagittal view. This image is of relatively poor contrast and has dimensions $512 \times 512$. This makes the image both a computationally demanding segmentation (as it has relatively large dimensions) and challenging in terms of accuracy.
The segmentation inputs and output can be seen in Figure \ref{fig:brain}. It can be seen that the sequential algorithm has perfomed reasonably in segmenting the white and grey matter and some of the brian stem. Considerable weighting had to be given to curvature in order to prevent leaks due to the poor contrast, resulting in a very rounded segmentation. 

\begin{table}[h]
\centering
\begin{tabular}{ | l | c | r | }
	\hline
	Algorithm Version  & Time (s)\\ \hline
  MATLAB 						 & 2737.84 \\
  C 								 & 322.81 \\
  CUDA (Unoptimized) & 44.68 \\
  CUDA (Optimized)   & 6.99  \\
  \hline
\end{tabular}\caption{Comparison of runtime for different algorithm versions - 2D brain segmentation}
\label{braintime}
\end{table}

The performance speedup attained on this larger image is therefore $46 \times$, which is greater than the speed up attained for the smaller $256 \times 256$ image. This motivates exploration into the effect of different image sizes on CUDA speed up, which is discussed in Section \ref{dimensions2d}. 

\subsubsection{Effect of Noise}

Denoising filters already exist as part of the CUDA SDK (i.e. \texttt{imageDenoising}). Our algorithm does not feature any image pre-processing algorithms such as denoising or blurring so its performance on noisy images is expected to be poor. In order to test this, artificial noise of 20\% and 40\% was added to the liver image as shown in Figure \ref{fig:livernoise}. 

\begin{figure}[h]
  \begin{center}
    \subfigure[Liver with 20\% Noise]{\label{fig:livernoise20}\includegraphics[scale=0.4]{images/livernoise20.PNG}}
    \subfigure[Segmentation with $T = 180, \epsilon = 55, \alpha = 0.003$]{\label{fig:livernoise20seg}\includegraphics[scale=0.4]{images/livernoise20seg.PNG}}\\
    \subfigure[Liver with 40\% Noise]{\label{fig:livernoise40}\includegraphics[scale=0.4]{images/livernoise40.PNG}}
    \subfigure[Segmentation with $T = 180, \epsilon = 75, \alpha = 0.003$]{\label{fig:livernoise40seg}\includegraphics[scale=0.4]{images/livernoise40seg.PNG}}
  \end{center}
  \caption{Segmentation of the liver with artifically added noise}
  \label{fig:livernoise}
\end{figure}

It can be seen that through manipulating the parameter $\epsilon$ segmentations are still approximately valid. The effect of noise on performance of the algorithm was negligable.

\subsubsection{Effect of Different Image Sizes}\label{dimensions2d}

It is found, as expected, that for all versions of the algorithm (CUDA and single threaded) compute time scales proportionally with the number of elements. For square 2D images, this implies that increasing image size by a factor of two in each dimension increases compute time by a factor of four. Figure \ref{fig:dimensions2d} displays compute times to 5000 iterations across the different algorithm versions. The same input image and mask was used for all test, with block dimensions of $32 \times 8$.

\begin{figure}[p]
	\begin{flushleft}
		\subfigure{\label{fig:dimensions2d}\includegraphics[scale=0.5]{images/dimensions2d.pdf}}
		\subfigure{\label{elements2d}\includegraphics[scale=0.5]{images/elements2d.pdf}}
	\end{flushleft}
	\label{fig:speed}\caption{Effect of different image sizes (a) On compute time (b) On number of elements computed per second]}
\end{figure}

It can be seen that the difference in compute time between the parallel and sequential versions is greatest for the largest images. This is due to the number of elements for the optimized CUDA version being considerably higher for all image sizes. Interestingly, Figure \ref{elements2d} shows that the number of elements computed per second is approximately constant for both the sequential and unoptimized CUDA implementations, but not for the optimized shared memory CUDA algorithm. This is due to the bottleneck being latency between device and shared memory and there not being enough occupancy of the GPU to mask this.




\subsection{3D Segmentations}

\begin{figure}[h]
  \begin{center}
    \subfigure{\includegraphics[scale=0.3]{images/3devolution0.PNG}}
    \subfigure{\includegraphics[scale=0.3]{images/3devolution1.PNG}}
    \subfigure{\includegraphics[scale=0.3]{images/3devolution2.PNG}}
    \subfigure{\includegraphics[scale=0.3]{images/3devolution3.PNG}}
  \end{center}
  \caption{Level Set Surface Evolution in 3D at 50, 200, 400 and 600 Iterations}
  \label{fig:evolution}
\end{figure}

Figure \ref{fig:evolution} illustrates a level set surface evolving in 3 dimensions. In order to visualise the level set evolving every certain number of iterations this the CUDA SDK example \texttt{volumeRender} was modified, however is not advised as doing so slows down the segmentation process by approximately an order of magnitude. This volume rendering engine uses ray tracing which is not advised for segmentation inspection, and therefore \textit{Paraview 3.4.0} (www.paraview.org) was used.

The 3D segmentation CUDA code uses a block size of $32 \times 4$. A block size of $32 \times 8$ causes the kernel invokation to fail due to the registers used per threads multiplied by the thread block size being greater $N$ (where for G80 NVIDIA hardware $N=8192$ 32-bit registers per multiprocessor. This limits the extent to which occupancy can be increased to mask latencies due to global memory loads. Section \ref{blocksizes} explores the effect of varying block sizes on performance.

\begin{figure}[p]
  \begin{center}
    \subfigure{\includegraphics[scale=0.55]{images/brain3d.PNG}}
    \subfigure{\includegraphics[scale=0.55]{images/brain3dseg3.PNG}}
    \subfigure{\includegraphics[scale=0.55]{images/brain3dseg1.PNG}}
    \subfigure{\includegraphics[scale=0.55]{images/brain3dseg2.PNG}}
  \end{center}
  \caption{Segmentation of a brain MRI dataset with parameters $T = 150, \epsilon = 50, \alpha = 0.03$ MRI data from \cite{brainweb}}
  \label{fig:brain3d}
\end{figure}
\begin{figure}[p]
  \begin{center}
    \subfigure[]{\includegraphics[scale=0.4]{images/heart3dseg.PNG}}
    \subfigure[]{\includegraphics[scale=0.4]{images/heart3dseg1.PNG}}
  \end{center}
  \caption{Segmentation of the right and left ventricles from a heart MRI dataset with parameters $T = 180, \epsilon = 60, \alpha = 0.02$ (a) Input data slice (b) Segmented heart clipped through $z$ plane}
  \label{fig:heart3d}
\end{figure}

As can be seen in Figure \ref{fig:brain3d} segmentation of the grey and white matter along with the brain stem is very anatomically detailed. This is due in part to the excellent quality of the \textit{BrainWeb} MRI data used (data is available from \cite{brainweb}). As the CFL condition had not been implemented in the 3D level set solver, it was found to converge at 1000 iterations and that $DT$ values greater than 0.1 resulted in instability.

The times taken to segment are shown in Figure \ref{brain3dtime}. MATLAB code is not shown as out of memory errors were encountered when loading such large arrays (even if these had not been encountered, the segmentation would have taken an infeasible amount of time).

\begin{table}[h]
\centering
\begin{tabular}{ | l | c | r | }
	\hline
	Algorithm Version  & Time (s)\\ \hline
  MATLAB 						 & N/A \\
  C 								 & 4697.5 \\
  CUDA (Unoptimized) & 392.8 \\
  CUDA (Optimized)   & 141.2  \\
  \hline
\end{tabular}\caption{Comparison of runtime for different algorithm versions - 3D Brain segmentation}
\label{brain3dtime}
\end{table}

An impressive speed up of $33 \times$ is observed. To further demonstrate the power of this algorithm testing was briefly done on a more mid-range 8800 GTX card, observing a speed up of $117 \times$ compared to the sequential algorithm.

In Figure \ref{fig:heart3d} segmentation of both the right and left ventricles can be seen. This segmentation data only had 17 $z$ plane slices (total resolution $256 \times 160 \times 17$).


\subsubsection{Effect of Different Volume Sizes}\label{dimensions3d}


Table \ref{table:dimensions3d} shows the effect of multiple volume sizes on the compute time to 1000 iterations on the optimized CUDA algorithm. Tests were not run on unoptimized or sequential code as results similar to those in 
As the CUDA optimized algorithm loops over the $k$ planes in a sequentially fashion, it is expected that doubling the volume size in the $z$ dimension would have an effect on compute time when compared to doubling the volume sizes in either the $x$ or $y$ dimensions. Therefore testing was done on non-cubic volume dimensions in order to explore this effect.

\begin{table}[h]
\centering
\begin{tabular}{ | c | c | c | }
	\hline
	Volume Dimensions  & Elements / Second & Time (s)\\ \hline
  $64 \times 64 \times 64  $	 & 54050.3			& 4.9 \\
  $128 \times 128 \times 128  $&	53676.8		& 39.1 \\
  $128 \times 256 \times 128  $& 53092.5   & 79.0 \\
  $128 \times 128 \times 256  $&	53485.1   & 78.4  \\
  $256 \times 256 \times 256  $&	52925.0   & 317.1  \\
  \hline
\end{tabular}\caption{Comparison of runtime for different volume sizes}
\label{table:dimensions3d}
\end{table}

In fact, it is found that there is no effect on performance which demonstrates the algorithms versatility. 

\subsubsection{Effect of Different CUDA Block/Grid Sizes}\label{blocksizes}

Finding the optimum block size for CUDA code is one of the most important ways to optimize performance. Block sizes should not be a user parameter (other than for testing purposes) as it assumed the developer would have chosen the optimum block size for maximum performance. Figure \ref{table:blocksizes} shows the compute times to 1000 iterations for the CUDA optimized code with different block sizes, all other parameters were held constant. 

\begin{table}[h]
\centering
\begin{tabular}{ | c | c | c | }
	\hline
	$BX \times BY$  & Threads/Block & Time (s) \\ \hline
  $32 \times 4$	 &128			& 141.7 \\
  $16 \times 8$	 &128			& 141.9 \\
  $16 \times 12$ &192 & 108.4 \\
  $32 \times 6$	 &192   & 107.4  \\
  $48 \times 4$	 &192  & 106.8  \\
  \hline
\end{tabular}\caption{Comparison of runtime for different block sizes}
\label{table:blocksizes}
\end{table}

It can be seen that for blocks with 192 threads performance is approximately constant across the different block arrangements. This is due to the fact that $BX$ has been chosen to be a multiple of 16 to maximise performance, the parameter $BY$ has much less of an effect on performance and should always be set secondary to $BX$.



\section{Discussion and Limitations}

\subsection{Speed}
This algorithm does not currently use a narrow band, introduced in Section \ref{upwinding}, to update the level set in either the sequential or parallel versions, making the algorithm more of a 'brute force' approach to segmentation. This is a feature to be included in the next version of the algorithm and has potential to further speed the algorithm by another order of magnitude.

Comparison of CPU and GPU code was done with algorithms that most closely mirrored each other. Although this standardizes the code, it does distort the results slightly as there is potential for optimizaton on the CPU by making effective use of the CPU cache, and multiple cores (if present). 

Goodman \cite{goodman} shows that CPU code may be slowed if a GPU kernel is executed and therefore suggests that CPU and GPU code run in seperate independent environments. To this effect, this has been catered for, increasing the accuracy of the speed up figures attained.

Furthermore, making comparisons with MATLAB code is inadvisable given the environment that MATLAB code runs. MATLAB code is JIT compiled, creating many problems when directly comparing compute times between the two versions. Speed ups were for this reason not measured against MATLAB code. In fact, the main purpose of the MATLAB code was to learn about the inner workings of level set segmentation.


\subsection{Accuracy}
The nature of segmenting images using thresholding and curvature terms favours segmentations of anatomical objects with a relatively homogenous gray value range. As the focus was on speed, a great deal of testing was not done on these forms of images.
	
Secondly, as discussed in Section \ref{3dvolumesegmentation} the current 3D level set segmentation solver does not integrate a reinitialisation algorithm. This is the largest limitation of this algorithm as it may result in instabilities in the level set function if $\nabla\phi$ values get too large. It should be noted that when implementing a 3D distance transform there is a major trade off between accuracy and speed.

Finally, there are currently no built in preprocessing filters. Denoising, blurring, sharpening and edge detection would most likely produce more accurate segmentations without affecting performance too greatly (provided CUDA kernels are used for these filters).




\chapter{Conclusions and Future Work}
\section{Conclusion}
In this project report a fast segmentation algorithm has been presented and analyzed. The implementation on the graphics device is very fast, with large two dimensional images and three dimensional volumes segmented $30$ to $40$ times faster (on a relatively low performance GPU) than sequential algorithms. 
The method of using level sets to segment images, and how to accelerate this process using GPUs has been discussed in great detail. The numerical methods used for the implementation were listed in Section \ref{levelsetalgorithm}. Giles' CUDA kernel for Laplace discretization in 3D \cite{mgiles} has been adapted for level set iteration. In Section \ref{results} it was seen that the power of GPU acceleration was demonstrated for very large data sets.

Given the wide range of applications level sets have in computing (image processing, computer graphics and physical simulation) this algorithm serves as an excellent framework to solve a diverse array of problems.
	
CUDA itself has been shown to be an excellent framework to accelerate computational problems in engineering, and is gaining more features and fewer limitations every few months. The principal disadvantages of CUDA are that it is only effective for very data parallel problems, and that it is not an industry standard. Recently, to counter the latter, it is very likely that it will in fact be replaced by \textit{OpenCL} (Open Computing Language). The syntax and architecture between CUDA and OpenCL will be very similar, allowing this code to be easily ported to OpenCL.

Nonetheless the impressive speedups attained using such low end hardware demonstrate the power of this parallel segmentation algorithm, and this makes segmentation with large 3D volumes much more practical in a clinical setting.

\section{Future Work}
There are several areas in which this algorithm could be extended. These revolve around three central themes of speed, accuracy and usability. 

In terms of speed, integrating the narrow band method into the algorithm will provide some further speed up, this however increases the complexity of the kernel potentially resulting in higher register usage and less occupancy. This, in combination to adding support for multiple GPUs and testing on very high performance hardware would be of significant interest. 

In terms of accuracy, it would be interesting, and reasonably straightforward, to integrate some of the already coded CUDA image processing examples (such as denoising, blurring, sharpening examples) to form a modular CUDA image processing and segmentation library that clinicians could work with. This could be included with a modular graphical user interface to make a very robust, usable and fast image processing library.

\begin{spacing}{1.2}
\nocite{3dfinitedifference}\nocite{brainweb}\nocite{chan2001acw}\nocite{cuda}\nocite{difi}\nocite{gist}\nocite{goodman}\nocite{gpgpudistance}\nocite{gui2005lse}\nocite{ibanez:isg}\nocite{kharlamov:id}\nocite{klar:igb}\nocite{Lefohn04astreaming}\nocite{mgiles}\nocite{narrowband}\nocite{nvidia2008cud}\nocite{osher2003lsm}\nocite{oshersethian}\nocite{rumpf2001lss}\nocite{sethian2003lsm}\nocite{sharma17cbl}\nocite{sparsefield}\nocite{nguyen2007gg}
\bibliographystyle{plain}
\bibliography{References}
\end{spacing}




\begin{spacing}{1.18}
\appendix
\chapter{MATLAB 2D Code}

\lstset{language=matlab,basicstyle=\footnotesize}
\lstinputlisting{code/simpleseg.m}

\chapter{CUDA 3D Kernel Source Code}
\lstset{language=c,basicstyle=\footnotesize}
\lstinputlisting{code/kernel.cu}

\newpage
\chapter{CUDA 3D Main Source Code}
\lstset{language=c,basicstyle=\footnotesize}
\lstinputlisting{code/main.cu}


\begin{center}
	Full versions of all the above source code, and their revisions, can be seen at\\
\texttt{cudaseg.googlecode.com}.
\end{center}

\includepdf[landscape]{images/riskassess.pdf}
\end{spacing}

\end{document}
